\documentclass{article}
\usepackage[utf8]{inputenc}
\usepackage{amsmath,amssymb,amsthm}
\usepackage{hyperref}

\newcommand{\G}{\mathbb{G}}
\newcommand{\F}{\mathbb{F}}

\title{Zarcanum technical notes}
\author{Cypher Stack\thanks{\url{https://cypherstack.com}}}
\date{\today}

\begin{document}
	
\maketitle

This document describes technical information relevant to the Zarcanum \cite{zarcanum} preprint\footnote{Specifically, version \texttt{20211108:135703}.} and its underlying algorithms and cryptography.
As with any review, it may be functionally incomplete, and may not reflect all errors or considerations that could arise with any particular implementation or application.
The author asserts no warranty and disclaims liability for its use.
The author further expresses no endorsement of the Zano project or its associated entities.


\section{Introduction}

Zarcanum is a set of protocols intended to enable staking operations in proof-of-stake blockchain assets that provide either amount confidentiality or, additionally, signer ambiguity.
The preprint describes two instantiations of such a protocol: one is intended for use in cases where only amount confidentiality is required, and the other for cases where signer ambiguity is additionally necessary.

The first Zarcanum protocol describes a method for staking a specified output.
This approach provides amount confidentiality, where the staker asserts that the specified output meets the consensus-specified difficulty requirement without revealing its value.
However, it does not provide source ambiguity; that is, the output being staked is explicitly specified in the staking operation.

The second Zarcanum protocol describes a method that extends confidential staking to additionally provide source amgiguity.
In this approach, a set of possible outputs is provided, and one output in the set is staked; amount confidentially is retained as well.


\section{Observations and findings}

In this section, we provide details on observations and findings from an analysis of the Zarcanum protocols.

\subsection{Security model}

The preprint describes confidential staking protocols that have complex interactions between cryptographic constructions and components, and for which parameter and design choices have nontrivial consequences.
For example, Section D describes different mitigations against a trivial staking condition bypass; one requires a particular set of proofs, while the other relies on reasoning about blockchain history.
As another example, Section E describes different mitigations against a brute-force attack by an adversarial sender; one requires careful selection of a scaling parameter in the staking condition, while the other uses a modification to the address and output formats.

While the preprint describes the components, methods, and operations used in the staking operations, and provides arguments for the security of such methods, it does not introduce a formal security model.
Instead, it provides a set of informal criteria and reasons against them.
Because comprehensive security assertions and proofs require a security model against which to reason, it is not possible to make definitive conclusions in this manner.

We strongly recommend extending the ideas and arguments in Zarcanum to a more comprehensive and formal security model, and then proving the security of the protocol and its components in this context.


\subsection{Proposition A.1 and Lemma 1}

We provide an alternate version of the results of Proposition A.1 and Lemma 1 from the preprint, as well as a proof that the resulting proving system is complete, special sound, and special honest-verifier zero knowledge.
This formalizes the ideas in the preprint that require the prover knows discrete logarithm witness elements.
These security properties may be useful for later integration within a broader formal protocol.

Let $\G$ be a finite cyclic group where the discrete logarithm problem is hard, and let $\F$ be its scalar field.
Let $0 \neq G,H \in \G$ be group elements with no efficiently-computable discrete logarithm relation.
We assume that $\G,G,H$ are implicit public parameters where needed.

Consider the relation
$$\mathcal{R}_1 = \{A,A' \in \G ; f,a \in \F : A = aH + fG, A' = fH + aG\}$$
and the following interactive protocol, which we will show is a sigma protocol for $\mathcal{R}_1$:
\begin{enumerate}
	\item The prover samples $r_0,r_1 \in \F$ uniformly at random.
	\item The prover computes $R_0 = r_0 (H + G)$ and $R_1 = r_1 (H - G)$, and sends these values to the verifier.
	\item The verifier samples nonzero $c \in \F$ uniformly at random, and sends this value to the prover.
	\item The prover computes $y_0 = r_0 + c(a + f)$ and $y_1 = r_1 + c(a - f)$, and sends these values to the verifier.
	\item The verifier accepts the proof if and only if the following hold:
		\begin{alignat}{1}
			y_0 (H + G) - c(A + A') &= R_0 \label{eqn:L1_1}\\
			y_1 (H - G) - c(A - A') &= R_1 \label{eqn:L1_2}
		\end{alignat}
\end{enumerate}
This protocol may be made non-interactive via the Fiat-Shamir technique, where the verifier challenge is replaced by a suitable transcript hash.
This technique further allows for binding arbitrary proof context into the transcript.

We show that this construction is complete, special sound, and special honest-verifier zero knowledge as a sigma protocol for the relation $\mathcal{R}_1$.

\begin{proof}
	Completeness follows trivially by inspection.

	We next show that the protocol is $2$-special sound by a standard rewinding argument, where we define an extractor that produces valid witness elements on accepting transcripts using distinct verifier challenges.
	Fix an initial transcript $(A,A',R_0,R_1)$, and let $c \neq c'$ be distinct verifier challenges for this transcript, with corresponding responses $(y_0,y_1)$ and $(y_0',y_1')$.
	We apply Equations \ref{eqn:L1_1} and \ref{eqn:L1_2} to these transcripts to obtain
	\begin{alignat*}{1}
		(y_0 - y_0')(H + G) - (c - c')(A + A') &= 0 \\
		(y_1 - y_1')(H - G) - (c - c')(A - A') &= 0
	\end{alignat*}
	and hence
	\begin{alignat}{1}
		\frac{y_0 - y_0'}{c - c'} (H + G) &= A + A'	\label{eqn:L1_alpha_0} \\	
		\frac{y_1 - y_1'}{c - c'} (H - G) &= A - A'. \label{eqn:L1_alpha_1}
	\end{alignat}
	Define $\alpha_0 = (y_0 - y_0')/(c - c')$ and $\alpha_1 = (y_1 - y_1')/(c - c')$, and note that both are well defined since $c \neq c'$.
	Adding and subtracting Equations \ref{eqn:L1_alpha_0} and \ref{eqn:L1_alpha_1}, we obtain the following expressions for $A$ and $A'$:
	\begin{alignat*}{1}
		A &= \frac{\alpha_0 + \alpha_1}{2} H + \frac{\alpha_0 - \alpha_1}{2} G \\
		A' &= \frac{\alpha_0 - \alpha_1}{2} H + \frac{\alpha_0 + \alpha_1}{2} G
	\end{alignat*}
	Hence we define $$a = \frac{\alpha_0 + \alpha_1}{2}$$ and $$f = \frac{\alpha_0 - \alpha_1}{2}$$ as the extracted witnesses, and the protocol is $2$-special sound.
	Observe that $a,f$ must be unique, or this implies a nontrivial discrete logarithm relationship between $G,H$.

	We finally show that the protocol is special honest-verifier zero knowledge.
	To do so, we define a simulator that, on a valid statement and uniformly-sampled verifier challenge, produces transcripts indistinguishable from those of real proofs.
	Fix a valid prover statement $(A,A')$, and sample a nonzero challenge $c \in \F$.
	The simulator samples random $y_0,y_1 \in \F$ and defines $R_0,R_1$ using Equations \ref{eqn:L1_1} and \ref{eqn:L1_2}, respectively.
	The resulting simulated proof will be accepted by an honest verifier.
	Because $G,H$ are independent generators, such a simulated proof is distributed identically to a real proof, and hence the protocol is special honest-verifier zero knowledge.

	This completes the proof.
\end{proof}


\subsection{Lemma 2}
\label{obs:lemma2}

We provide an alternate version of the result of Lemma 2 from the preprint, as well as a proof that the resulting proving system is complete, special sound, and special honest-verifier zero knowledge.
This formalizes the ideas in the preprint that require the prover knows discrete logarithm witness elements.
These security properties may be useful for later integration within a broader formal protocol.

Let $\G$ be a finite cyclic group where the discrete logarithm problem is hard, and let $\F$ be its scalar field.
Let $0 \neq G,H,X \in \G$ be group elements with no efficiently-computable discrete logarithm relation.
We assume that $\G,G,H,X$ are implicit public parameters where needed.

Consider the relation
$$\mathcal{R}_2 = \{C,C' \in \G ; f,a,x,x' \in \F : C = aH + fG + xX, C' = fH + aG + x'X\}$$
and the following interactive protocol, which we will show is a sigma protocol for $\mathcal{R}_2$:
\begin{enumerate}
	\item The prover samples $r_0,r_1,s_0,s_1 \in \F$ uniformly at random.
	\item The prover computes $R_0 = r_0 (H + G) + s_0 X$ and $R_1 = r_1 (H - G) + s_1 X$, and sends these values to the verifier.
	\item The verifier samples nonzero $c \in \F$ uniformly at random, and sends this value to the prover.
	\item The prover computes and sends these values to the verifier:
		\begin{alignat*}{1}
			y_0 &= r_0 + c(a + f) \\
			y_1 &= r_1 + c(a - f) \\
			z_0 &= s_0 + c(x + x') \\
			z_1 &= s_1 + c(x - x')
		\end{alignat*}
	\item The verifier accepts the proof if and only if the following hold:
		\begin{alignat}{1}
			y_0 (H + G) + z_0 X - c(C + C') &= R_0 \label{eqn:L2_1}\\
			y_1 (H - G) + z_1 X - c(C - C') &= R_1 \label{eqn:L2_2}
		\end{alignat}
\end{enumerate}
This protocol may be made non-interactive via the Fiat-Shamir technique, where the verifier challenge is replaced by a suitable transcript hash.
This technique further allows for binding arbitrary proof context into the transcript.

We show that this construction is complete, special sound, and special honest-verifier zero knowledge as a sigma protocol for the relation $\mathcal{R}_2$.

\begin{proof}
	Completeness follows trivially by inspection.

	We next show that the protocol is $2$-special sound by a standard rewinding argument, where we define an extractor that produces valid witness elements on accepting transcripts using distinct verifier challenges.
	Fix an initial transcript $(C,C',R_0,R_1)$, and let $c \neq c'$ be distinct verifier challenges for this transcript, with corresponding responses $(y_0,y_1,z_0,z_1)$ and $(y_0',y_1',z_0',z_1')$.
	We apply Equations \ref{eqn:L2_1} and \ref{eqn:L2_2} to these transcripts to obtain
	\begin{alignat*}{1}
		(y_0 - y_0')(H + G) + (z_0 - z_0')X - (c - c')(C + C') &= 0 \\
		(y_1 - y_1')(H - G) + (z_1 - z_1')X - (c - c')(C - C') &= 0
	\end{alignat*}
	and hence
	\begin{alignat}{1}
		\frac{y_0 - y_0'}{c - c'} (H + G) + \frac{z_0 - z_0'}{c - c'} X &= C + C'	\label{eqn:A1_alpha_0} \\	
		\frac{y_1 - y_1'}{c - c'} (H - G) + \frac{z_1 - z_1'}{c - c'} X &= C - C'. \label{eqn:A1_alpha_1}
	\end{alignat}
	Define the following:
	\begin{alignat*}{1}
		\alpha_0 &= \frac{y_0 - y_0'}{c - c'} \\
		\alpha_1 &= \frac{y_1 - y_1'}{c - c'} \\
		\beta_0 &= \frac{z_0 - z_0'}{c - c'} \\
		\beta_1 &= \frac{z_1 - z_1'}{c - c'}
	\end{alignat*}
	Note that all are well defined since $c \neq c'$.
	Adding and subtracting Equations \ref{eqn:A1_alpha_0} and \ref{eqn:A1_alpha_1}, we obtain the following expressions for $C$ and $C'$:
	\begin{alignat*}{1}
		C &= \frac{\alpha_0 + \alpha_1}{2} H + \frac{\alpha_0 - \alpha_1}{2} G + \frac{\beta_0 + \beta_1}{2} X \\
		C' &= \frac{\alpha_0 - \alpha_1}{2} H + \frac{\alpha_0 + \alpha_1}{2} G + \frac{\beta_0 - \beta_1}{2} X
	\end{alignat*}
	Hence we define
	\begin{alignat*}{1}
		a &= \frac{\alpha_0 + \alpha_1}{2} \\
		f &= \frac{\alpha_0 - \alpha_1}{2} \\
		x &= \frac{\beta_0 + \beta_1}{2} \\
		x' &= \frac{\beta_0 - \beta_1}{2}
	\end{alignat*}
	as the extracted witnesses, and the protocol is $2$-special sound.
	Observe that $a,f,x,x'$ must be unique, or this implies a nontrivial discrete logarithm relationship between $G,H,X$.

	We finally show that the protocol is special honest-verifier zero knowledge.
	To do so, we define a simulator that, on a valid statement and uniformly-sampled verifier challenge, produces transcripts indistinguishable from those of real proofs.
	Fix a valid prover statement $(C,C')$, and sample a nonzero challenge $c \in \F$.
	The simulator samples random $y_0,y_1,z_0,z_1 \in \F$ and defines $R_0,R_1$ using Equations \ref{eqn:L2_1} and \ref{eqn:L2_2}, respectively.
	The resulting simulated proof will be accepted by an honest verifier.
	Because $G,H,X$ are independent generators, such a simulated proof is distributed identically to a real proof, and hence the protocol is special honest-verifier zero knowledge.

	This completes the proof.
\end{proof}


\subsection{Ring signature extension}

Section 6.1 of the preprint describes a method to extend MLSAG \cite{ringct} signature dimensionality to account for the addition of an ``extended commitment'' $C$ to the same amount as a standard output commitment $A$ and group element $Q$, such that a prover knows some $x \in \F$ such that $C - A - Q = xX$ for a fixed global generator $X \in \G$, and also knows some $q \in \F$ such that $Q = qG$ for a fixed global generator $G \in \G$.
Further, this proof must be ambiguous in the sense that each tuple $(A, Q)$ is contained within a larger cover set.

We note that MLSAG signatures are not proofs of knowledge in the sense that their construction does not appear to readily admit an extractor for the required discrete logarithm witnesses.
However, their use in confidential transaction protocols admits useful security properties relating to multidimensional linkable ring signatures for linkability, anonymity, unforgeability, and the like.
While the MLSAG extension described here would seem to inherit these properties, it is not immediately clear if they are sufficient for the broader Zarcanum protocol, or if a later formal security model would require other properties provided by zero-knowledge proofs of knowledge.

Should a broader protocol's security model require that this modified construction be a zero-knowledge proof of knowledge or otherwise require properties like special soundness or special honest-verifier zero knowledge, it is possible to use other constructions like Triptych \cite{triptych} that provide the same multidimensional linkable ring signature functionality.
However, this comes at the cost of a linking tag definition that differs from the linear key image construction used in MLSAG-based protocols, and may not be suitable for this reason.

Additionally, the use of MLSAG linkable ring signatures in the protocol for staking and spend operations requires careful consideration to ensure that source information is not linked between a staking operation and a later spend.
Existing security models for linkable (and multidimensional linkable) ring signature constructions like MLSAG or CLSAG \cite{clsag} do not specifically address the situation where related signature algorithms (using, in this case, different dimensionality) are used concurrently within a broader protocol.
The use of a suitable zero-knowledge proof of knowledge instead of a witness-indistinguishable construction is likely to minimize this type of risk.


\subsection{Range proof extension}
\label{obs:range}

Section 5.1 of the preprint discusses ways to show that a double-blinded Pedersen commitment is to a value in the range $[0,2^n)$ for some fixed natural number $n$.
It mentions one method is to extend a range proving system like Bulletproofs \cite{bp} or Bulletproofs+ \cite{bp_plus} to account for the added blinding factor.

We note that such an extension for Bulletproofs is straightforward, and is described in a version of the preprint for the Lelantus \cite{lelantus} transaction protocol.
It is likely possible to produce such an extension for the more efficient Bulletproofs+ protocol, but we do not know of any work that has yet done so.


\subsection{Linear composition proof}

Section A.2 of the preprint describes a ``linear composition proof'' intended to show the existence of a representation of a given group element with respect to two specific generators.
This proof is used in several places within the Zarcanum protocol:
\begin{enumerate}
	\item\label{item:mirror} To prove (in step 5 of Section 6.2) that $C$ and $C'$ are mirror commitments.
	\item\label{item:range} To prove (in step 8 of Section 6.2) the validity of the relationship between $E$ and $B$.
	\item To prove (in Section 6.1) the validity of the relationship between $W$ and $C$.
	\item To prove (in Section D.1) that $f + q \neq 0$.
\end{enumerate}

The use of the alternate version of Lemma 2 presented in Observation \ref{obs:lemma2} removes the need for a linear composition proof in item \ref{item:mirror} above.
Further, we note in Observation \ref{obs:range} that the Bulletproofs range proving system (and likely the Bulletproofs+ range proving system) can be used directly with double-blinded commitments, which would remove the need for a linear composition proof in item \ref{item:mirror}.

The presentation of the proof in Section A.2 is somewhat informal.
No conditions are placed on the generators $A,B$ that may be needed for the use cases listed; for example, it may be required that the generators have no efficiently-computable discrete logarithm relationship in order to extract unique witnesses from successful verification.
Further, the given protocol states that successful verification convinces the verifier that a representation of $C$ with respect to the generators $A,B$ exists, but makes no particular claim that the prover knows this representation.
Indeed, since $\G$ is cyclic, such a representation exists for every $C \in \G$.

It may be preferable to define the linear composition proving system more formally, such that it is a zero-knowledge proof of knowledge, and that the resulting representation is unique (for a computationally-bound prover).
Computational uniqueness of the representation, as well as being zero knowledge, requires in particular that $A,B$ have no efficiently-computable discrete logarithm relationship; fortunately, this requirement is satisfied by the generators used in the required use cases.
Note that soundness does not require this constraint.
The proving system may be modified slightly to accommodate a straightforward security proof, which we describe here for completeness.

Let $\G$ be a finite cyclic group where the discrete logarithm problem is hard, and let $\F$ be its scalar field.
Let $0 \neq A,B \in \G$ be group elements with no efficiently-computable discrete logarithm relation.
We assume that $\G,A,B$ are implicit public parameters where needed.

Consider the relation
$$\mathcal{R}_3 = \{C \in \G; a,b \in \F : C = aA + bB\}$$
and the following interactive protocol, which is a sigma protocol for $\mathcal{R}_3$:
\begin{enumerate}
	\item The prover samples $r_0, r_1 \in \F$ uniformly at random.
	\item The prover computes $R = r_0 A + r_1 B$, and sends this value to the verifier.
	\item The verifier samples nonzero $c \in \F$ uniformly at random, and sends this value to the prover.
	\item The prover computes $y_0 = r_0 + ca$ and $y_1 = r_1 + cb$, and sends these values to the verifier.
	\item The verifier accepts the proof if and only if
	\begin{equation}
		\label{eqn:representation}
		y_0 A + y_1 B - cC = R.
	\end{equation}
\end{enumerate}
This protocol may be made non-interactive via the Fiat-Shamir technique, where the verifier challenge is replaced by a suitable transcript hash.
This technique further allows for binding arbitrary proof context into the transcript.

We show that this construction is complete, special sound, and special honest-verifier zero knowledge as a sigma protocol for the relation $\mathcal{R}_3$.

\begin{proof}
	Completeness follows trivially by inspection.

	We next show that the protocol is $2$-special sound by a standard rewinding argument, where we define an extractor that produces valid witness elements on accepting transcript using distinct verifier challenges.
	Fix an initial transcript $(C,R)$, and let $c \neq c'$ be distinct verifier challenges for this transcript, with corresponding responses $(y_0, y_1)$ and $(y_0', y_1')$.
	We apply Equation \ref{eqn:representation} to these transcripts to obtain
	$$(y_0 - y_0')A + (y_1 - y_1')B - (c - c')C = 0$$
	and hence
	$$\frac{y_0 - y_0'}{c - c'} A + \frac{y_1 - y_1'}{c - c'} B = C.$$
	Define $$a = \frac{y_0 - y_0'}{c - c'}$$ and $$b = \frac{y_1 - y_1'}{c - c'}$$ as the extracted witnesses, which are well defined since $c \neq c'$, and the protocol is $2$-special sound.
	Observe that $a,b$ must be unique, or this implies a nontrivial discrete logarithm relationship between $A,B$.

	We finally show that the protocol is special honest-verifier zero knowledge.
	To do so, we define a simulator that, on a valid statement and uniformly-sampled verifier challenge, produces transcripts indistinguishable from those of real proofs.
	Fix a valid prover statement $C$, and sample a nonzero challenge $c \in \F$.
	The simulator samples random $y_0,y_1 \in \F$ and defines $R$ using Equation \ref{eqn:representation},
	The resulting simulated proof will be accepted by an honest verifier.
	Because $A,B$ are independent generators, such a simulated proof is distributed identically to a real proof, and hence the protocol is special honest-verifier zero knowledge.

	This completes the proof.
\end{proof}


\bibliographystyle{plain}
\bibliography{main}

\end{document}
